\section*{Use Cases}

\prob{3} Many networked systems and applications now rely on machine learning
for performance modeling and prediction because modeling with closed-form
equations has become too complex (and inaccurate). Give an example of a
networked system---and associated {\em performance} prediction problem---where
machine learning has been useful because closed-form modeling is not feasible.

\answerbox{1}{Predicting web search response time is one example that we
discussed in class. There are many possible answers.}
\eprob

\prob{4} We discussed applications of machine learning to security, such as
detecting Internet scanning. Scanning for a web vulnerability
(e.g., the Log4j vulnerability we explored in class) might look different than
web traffic. (1) Why might a scan look different than regular Web traffic?
(2) What is one feature you could encode in a machine learning model that
could distinguish scanning from regular web traffic?

\answerbox{1}{1. Scans are looking for vulnerabilities and thus do not need
to complete a full transaction. 2. Request rate (there are other
possibilities, if justified!).}
\eprob

\section*{Data Acquisition}

\prob{2} Suppose you want to extract all Netflix traffic from a traffic
capture.  Capturing all traffic to and from the IP address for {\tt
netflix.com}  will yield all Netflix traffic streams.
\framebox{
\yesnono
}
\eprob

\prob{2}
In class, we used the domain name system (DNS) lookup traffic to identify Netflix
traffic. This approach can work in practice but is imperfect. List one reason
why DNS names may not always be practical for identifying traffic for services
like Netflix. {\bf (Answer box on next page!)}

\answerbox{1}{Domain names can change over time.
DNS traffic is becoming increasingly encrypted, making it difficult to see
domain name lookups and responses. Another reason is that the DNS names to
these services can change.}
\eprob

\prob{5}
What are the five header types in a network packet that make up a flow?

\answerbox{1}{Source and destination IP address, source and destination port,
protocol.}
\eprob

\prob{3}
List three advantages to active Internet measurement over passive Internet measurement.

\answerbox{1}{Direct measurement of desired effect, timing and frequency can
be controlled, little to no privacy risks. (There was a whole slide on this in
the board notes that we came up with in class, so anything from that slide, or
anything reasonable, will suffice.)}
\eprob


\section*{Feature Engineering}

\prob{3}
Features should be characteristic of fundamental differences between classes, rather than
simply characteristics of the dataset. Suppose you have a {\em single} packet
trace from the University of Chicago campus network, where Log4j scans are
being conducted at the same time as regular traffic.
You decide to use {\em only incoming network traffic} to train a detection model, using
features that include all of five the fields for incoming network
traffic, and the detection model works really well. But, when your friends at
Northwestern try
to use your model, it doesn't work at all. What feature or features might be at fault,
{\bf and why}?

\answerbox{0.75}{Destination IP address, because the destination IP addresses
will be different at Northwestern. (Other fields, like source port, may
be an acceptable answer if well-explained.)}
\eprob

\section*{Feedback}
\prob{1}
Interest (1=Boring!; 10=Amazing!):
\shortanswerbox{0.5}{5}
Difficulty (1=Too easy; 10=Too hard):
\shortanswerbox{0.5}{5}
\eprob
\prob{1}
1. One thing you like. 2. One suggestion for improvement:

\answerbox{0.75}{More free food.}
\eprob

